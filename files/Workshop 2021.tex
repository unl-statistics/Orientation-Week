% LaTeX Article Template - using defaults
\documentclass{article}

\setlength{\oddsidemargin}{-.25in}
\setlength{\topmargin}{-.25in}
\setlength{\textwidth}{6.75in}
\setlength{\textheight}{8.5in}

\thispagestyle{empty}

\begin{document}

\begin{center}
{\large \textbf{STAT 218 Workshop:  Fall, 2021}}
\end{center}

\begin{itemize}
\item \textbf{Monday, 12:00-2:00}:  Large Conference Room -- New TAs
\begin{itemize}
\item Lunch! (12:00-12:45)
\item Describe STAT 218 (12:45-1:30) -- Erin and Alison
\begin{itemize}
\item What we expect from the mentor TAs
\item What we expect from the new TAs
\item Constructive feedback:  both directions
\item Contract:  by the end of week 1, the instructional team should have a written contract detailing expectations (homework turnaround times, office hours, meeting times, etc)
\item Lesson plans: These will be collected in a portfolio, and will give you a ready-to-go resource for when you begin teaching in Spring, 2022.  Begin with a class schedule and syllabus.  By Mid-Term, we expect that will you will have the first three weeks of class ready to go.  By the end of the semester, a rough draft for the rest of the semester. Lab/grader vs own section?
\item Teaching journal.  We'll give you assignments for journaling on occasion.  Also, this is great place to keep track of what worked in your faculty member/mentor TA's class, and what didn't.  What will you incorporate (steal?) for next semester, and what will you change?  First:  Take five minutes to write a bit about what you're anxious about/most excited about/questions you want answered during the workhop
\item Those not teaching working with STAT 218
\item STAT 892
\end{itemize}
\item Course Philosophy (1:30-2:00) -- Erin (Alison jump in)
\begin{itemize}
\item Statistical Literacy over Statistical Arithmetic
\item STAT 218 as ACE course--supporting area are writing and critical thinking
\item Philosophy for other courses
\end{itemize}
\item Hand out JSDSE article  Homework:  Read by Wednesday
\item Hand out Microteaching assignments
\item Hand out GAISE College Report: start reading--discuss in STAT 892
\item Department Meeting (3:00-4:30) -- Hardin Hall Auditorium
\end{itemize}
\item \textbf{Tuesday, 8:00 - 12:30}:   Grad Studies TA Workshop, City Union
\item \textbf{Wednesday, 9:00 - 11:00}: 49 Hardin Hall -- New STAT TAs + All STAT 218 TAs
\begin{itemize}
\item Important Concepts (9:00-9:20)  Star the concepts you think about big ideas (things you would be embarrassed about if your students didn't remember a month after the class was over).  Circle the concepts you think should be taught in the class.  Cross-out the things you think have no business being taught in a STAT 218 class. 
\item Discussion of Key Concepts (9:20-10:00) -- Erin (Alison jump in)
\begin{itemize}
\item Have groups discuss the concepts they selected.  Can they narrow down?
\item Summarize on board
\item Hand out ``official'' list of important concepts.
\item What concepts do you think students have the most trouble with? Why? What do you think we can do to help facilitate understanding?
\item Questions?
\end{itemize}
\item Break (10:00-10:15)
\item Data Viz Discussion/Writing (10:15-11:00) -- Erin and Alison; Frame as big picture; how to communicate big ideas to students; connection between stat literacy/critical thinking/writing
\begin{itemize}
\item Writing Activity:  What were your initial thoughts about the Data Viz article?  Write for a few minutes about what you took away from this article. 
\item Get in groups and discuss your writing; have groups share their thoughts with the larger group
\item Why we think writing is integral to statistics/Intro to STAT 892
\item Writing not as a separate/additional subject, but as a way to enhance learning of statistics
\item One thing you'll find as you go through STAT 218 is that you gain a deeper understanding of elementary statistics (ANY statistics, really) through teaching it.  We believe the same is true for your students--they will gain a deeper understanding of the STAT 218/other concepts by writing about them.
\item Writing Activity:  Write for a few minutes about what kinds of alternative assessments you think work well in STAT 218. Maybe it's an assessment you experienced as a student. 
\item Trade papers with partner; compare alternative assessments; discuss as larger group
\item Students often come into STAT 218 assuming that it will be a ``math'' class, and are sometimes angry/confused to find it is not.  Writing (from the very beginning) and emphasizing inference from the start of the class are two ways to make clear from the outset that statistics is not math.
\item Another Writing Activity:  Write for a few minutes about how you think the STAT 218 class can be structured to make clear to the students that statistics is not math.
\item Discuss in groups; share with larger group
\item All of these writing exercises are examples of Low Stakes Writing. They had different goals.  The first was Writing to Probe a Subject--basically to get you to think about the article.  The second and third were tools for us to gauge/guide future discussion.  
\item Incorporating writing into other things we'll discuss: activities, assessments, important concepts
\item What we'll be discussing in STAT 892
\begin{itemize}
\item  writing/thinking problems -- how you can use student writing to gauge understanding
\item low, mid and high stakes writing tasks
\item peer review 
\item constructing assignments
\item assessing writing/understanding
\end{itemize}
\end{itemize}

\item Break (11:00-11:30)

\item Wiley Plus Training (11:30-2:00)
\begin{itemize}
\item Lunch! (11:30-12:30)
\item Wiley Plus Training (12:30-2)
\end{itemize}
\item Break (2:00-2:15)
\item Assessment (2:15-3:30) -- Erin and Alison
\begin{itemize}
\item Homework and alternatives to homework
\item How writing fits in
\item Input from experienced TAs
\item Grading practice
\end{itemize}
\end{itemize}

\item \textbf{Thursday, 9:00 - 11:00}: 49 Hardin Hall -- New STAT TAs + All STAT 218 TAs

\begin{itemize}
\item Peer Review of syllabi (9:00-10:00) -- Erin
\begin{itemize}
\item write a quick author's note for your syllabus--what elements are you most unsure about?
\item review each person's syllabus in your group 
\item how peer review can be used in STAT 218/other
\item CASNR syllabus policy
\item discuss as whole group
\end{itemize}
\item Break (10:00-10:10)
\item Lessons learned from last year (10:10-11:00) -- Erin and Alison
\begin{itemize}
\item Anything that went really well that you'd like to share?
\item Anything you'd like suggestions on to improve?
\item Questions/concerns for Fall, 2021?
\end{itemize}
\item Break (11-12:30; lunch on your own)
\end{itemize}
\item \textbf{Thursday, 12:30 - 3:00}: 49 Hardin Hall -- ALL TAs
\begin{itemize}
\item University Policies -- Erin with general discussion
\begin{itemize}
\item Services for Students with Disabilities (SSD)
\item Counseling and Psychological Services (CAPS)
\item Canvas
\item Emergency Procedures
\item Academic Integrity (UNL and Departmental policies)
\item COVID policies
\end{itemize}
\end{itemize}

\item \textbf{Friday, 10:00-12:30}: 49 Hardin Hall -- New STAT TAs + All STAT 218 TAs
\begin{itemize}
\item Activities (10:00-11:15) -- Erin and Alison
\begin{itemize}
\item Before you carry out an activity:  what is its purpose?  what do you want the students to get out of the activity?  
\item Demo activity: Babies
\item Tips from returning TAs on activities/labs
\item Active learning means active thinking--not necessarily ``hands-on''
\item Prop check-out sheet
\end{itemize}
\item Break (11:15-11:30)
\item Meet with your mentor (11:30-12:30)
\item Break (12:30-2:00; lunch on your own)
\end{itemize}

\item \textbf{Friday, 2:00-3:30}: 49 Hardin Hall -- New STAT TAs
\begin{itemize}
\item Microteaching (2:00-3:15) -  new TAs (Aaron, Drew, Tyler, Wyatt, Jack, Jayden, Huy, Susweta)
\item Q \& A (3:15-3:30)
\item Assignment for STAT 892 (due first class period):  Write a question to assess students' understanding of your microteaching topic. The question should NOT emphasize calculation. Begin reading the GAISE recommendations.
\end{itemize}

\end{itemize}
\end{document}
