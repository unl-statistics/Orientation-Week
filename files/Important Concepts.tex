% LaTeX Article Template - using defaults
\documentclass{article}

\setlength{\oddsidemargin}{-.25in}
\setlength{\topmargin}{-.25in}
\setlength{\textwidth}{6.75in}
\setlength{\textheight}{8.5in}


\input{latex.def}


\thispagestyle{empty}

\begin{document}

\begin{center}
{\large \textbf{STAT 218 Learning Objectives}}
\end{center}

\begin{itemize}
\item \large{Big Picture Goals}
\normalsize
\begin{itemize}
\item[12.] Random sampling allows results of surveys and experiments to be extended to the population from which the sample was taken 
\item[13.] Variability is natural, predictable, and quantifiable 
\item[19.] How to interpret numerical summaries and graphical displays of data �- both to answer questions and to check conditions (to use statistical procedures correctly)
\item[29.] Believe and understand why association is not causation 
\item[30.] Believe and understand why random assignment in comparative experiments allows cause-and-effect conclusions to be drawn 
\item[34.] The concept of a sampling distribution and how it applies to making statistical inferences based on samples of data (including the idea of standard error)
\item[55.] The concept of statistical significance, including significance levels and p-values  
\item[57.] How to interpret statistical results in context 
\item[58.] How to critique news stories and journal articles that include statistical information, including identifying what�s missing in the presentation and the flaws in the studies or methods used to generate the information 
\item[61.] How to determine the population to which the results of statistical inference can be extended, if any
\item[62.] Students should believe and understand why data beat anecdotes
\item[80.] Understand how simulation may be used to determine strength of evidence
\item[89.] Translating a research question/scenario into a null model
\end{itemize}

\item \large{In achieving these big picture goals, you should cover the following topics}
\normalsize
\begin{itemize}
\item[1.] Quantitative versus categorical variables
\item[5.] How to interpret histograms
\item[7.] Skewed and symmetric distributions
\item[8.] Measures of center (mean and median)
\item[9.] Sampling strategies:  at least that not everything is a SRS
\item[14.] How to calculate variance and standard deviation (with caveats)
\item[15.] Characteristics of the normal distribution? Depends on how your class is structured
\item[23.] Distinguish between explanatory and response variables
\item[25.] Use scatterplots to identify patterns and outliers
\item[26.] Know what a line of best fit represents and how to use it to make predictions
\item[31.] Describe the sampling distribution of a statistic and define the standard error of a statistic
\item[35.] How to calculate (obtain?) confidence intervals for the population proportion, $p$
\item[36.] How to calculate (obtain?) confidence intervals for the population mean, $\mu$
\item[37.] Interpret the result of a confidence interval in the context of the problem
\item[39.] Elementary probability rules
\item[40.] Probability rules for complements (necessary only if using normal distribution)
\item[42.] Addition rule (necessary only if using normal distribution)
\item[47.] Describe how changing the sample size and/or the confidence level will affect the width of the confidence interval
\item[48.] Given a study objective, choose appropriate null and alternative hypotheses, including
determining whether the alternative should be one-sided or two-sided
\item[49.] Given a study and p-value, explain in context that p-value is a probability of getting a
sample statistic as extreme or more extreme than what was seen in the sample given that
the null hypothesis is true
\item[51.] Given a study, interpret the results of a test of significance in context
\item[52.] Given a study objective, significance level ($\alpha$) and summary statistics, understand the steps involved in conducting a formal
test of significance on a population mean (or a population proportion).
\item[59.] How to communicate the results of a statistical analysis 
\item[60.] How to appropriately use statistical inference to answer research questions 
\item[63.] Understand the difference between matched-pairs and two-sample data (if time allows)
\item[69.] Understand the importance of controlling for sources of extraneous variation in studies
\item[72.] The importance of examining graphs in describing a data set
\item[76.] Describing relations in a two-way table
\item[81.] Understand what a null model represents
\item[82.] Assumptions necessary for inference via simulation versus assumptions necessary for inference via traditional methods
\item[83.] Determining which analysis method is appropriate for a given scenario
\item[87.] Understand the relationship between hypothesis testing and confidence intervals
\item[88.] Understand that not all outcomes of a random experiment are equally likely
\item[90.] Representing null models in both words and symbols
\end{itemize}

\item \large{No-Nos}
\normalsize
\begin{itemize}
\item[15.] What happens to mean and variance when the unit of measurement is changed
\item[17.] Normal quantile plots
\item[20.] How to calculate the correlation (by  hand)
\item[21.] How to calculate the estimated slope and intercept for lines of best fit (by hand)
\item[45.] Bayes' Theorem
\item[68.] Understand Simpson's paradox
\item[78.] The $F$ distribution
\item[86.] Understand that $n>30$ ensures the Central Limit Theorem holds
\end{itemize}

\end{itemize}
\end{document}
