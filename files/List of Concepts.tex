% LaTeX Article Template - using defaults
\documentclass{article}

\setlength{\oddsidemargin}{-.25in}
\setlength{\topmargin}{-.25in}
\setlength{\textwidth}{6.75in}
\setlength{\textheight}{8.5in}


\input{latex.def}


\thispagestyle{empty}

\begin{document}

\begin{center}
{\large \textbf{STAT 218 Learning Objectives}}
\end{center}

\begin{enumerate}

\item Quantitative versus categorical variables
\item Pie charts
\item Stem-and-leaf plots
\item How to construct histograms
\item How to interpret histograms
\item Relative frequency
\item Skewed and symmetric distributions
\item Measures of center (mean and median)
\item Sampling strategies
\item Understand the impact of unusual observations
\item Given a histogram, be able to determine the approximate location of the median and quartiles
\item Random sampling allows results of surveys and experiments to be extended to the population from which the sample was taken
\item Variability is natural, predictable and quantifiable
\item How to calculate variance and standard deviation
\item What happens to mean and variance when the unit of measurement is changed
\item Characteristics of the normal distribution
\item Normal quantile plots
\item Using the normal table to find probabilities
\item How to interpret numerical summaries and graphical displays of data -- both to answer questions and to check conditions (to use statistical procedures correctly)
\item How to calculate the correlation
\item How to calculate the estimated slope and intercept for lines of best fit
\item The relationship between correlation and estimated slope
\item Distinguish between explanatory and response variables
\item Make a scatterplot
\item Use scatterplots to identify patterns and outliers
\item Know what a line of best fit represents and how to use it to make predictions
\item Interpret the value of the squared correlation coefficient
\item Calculate residuals
\item Believe and understand why association is not causation
\item Believe and understand why random assignment in comparative experiments allows cause-and-effect conclusions to be drawn
\item Describe the sampling distribution of a statistic and define the standard error of a statistic
\item Describe the sampling distribution of $\wb X$
\item Describe the sampling distribution of $\wh p$
\item The concept of a sampling distribution and how it applies to making statistical inferences based on samples of data (including the idea of standard error)
\item How to calculate confidence intervals for the population proportion, $\pi$
\item How to calculate confidence intervals for the population mean, $\mu$
\item Interpret the result of a confidence interval in the context of the problem
\item Explain how inferences based on the $t$-distribution are robust
\item Elementary probability rules
\item Probability rules for complements
\item Multiplication rule
\item Addition rule
\item Conditional probability
\item Mutually exclusive and independent events
\item Bayes' Theorem
\item Determine appropriate degrees of freedom
\item Describe how changing the sample size and/or the confidence level will affect the width of the confidence interval
\item Given a study objective, choose appropriate null and alternative hypotheses, including determining whether the alternative should be one-sided or two-sided
\item Given a study and p-value, explain in context that p-value is a probability of getting a sample statistic as extreme or more extreme than what was seen in the sample given that the null hypothesis is true
\item Given a test statistic, calculate a p-value based on the standard normal distribution or $t$-distribution as appropriate
\item Given a study, interpret the results of a test of significance in context
\item Given a study objective, significance level $(\alpha)$ and summary statistics, understand the steps involved in conducting a formal test of significance on a population mean (or a population proportion)
\item Explain the relationship between a confidence interval and a two-sided hypothesis test
\item Given results from a hypothesis test, comment on the impact of sample size and the practical importance
\item The concept of statistical significance, including significance levels and p-values
\item The concept of confidence interval, including the interpretation of confidence level and margin of error
\item How to interpret statistical results in context
\item How to critique news stories and journal articles that include statistical information, including identifying what's missing in the presentation and the flaws in the the studies or methods used to generate the information
\item How to communicate the results of a statistical analysis
\item How to appropriately use statistical inference to answer research questions
\item How to determine the population to which the results of statistical inference can be extended, if any
\item Students should believe and understand why data beat anecdotes
\item Understand the difference between matched-pairs and two-sample data
\item Describe the sampling distribution of the difference between two means as specifically as possible
\item Conduct statistical inference based on matched-pairs data
\item Conduct statistical inference based on two-sample data
\item Define the placebo effect and explain the purpose of a placebo
\item Understand Simpson's paradox
\item Understand the importance of controlling for sources of extraneous variation in studies
\item Explain the advantages of using a double-blind experiment
\item Explain the advantages of using blocking
\item The importance of examining graphs in describing a data set
\item Sample spaces in probability models
\item The binomial distribution
\item Power and Type II error
\item Describing relations in two-way tables
\item Pooled vs unpooled variance in two-sample $t$-tests
\item The $F$ distribution
\item Critical values in significance testing
\item Understand how simulation may be used to determine strength of evidence
\item Understand what a null model represents
\item Assumptions necessary for inference via simulation versus assumptions necessary for inference via traditional methods
\item Determining which analysis method is appropriate for a given scenario
\item Simulated standard error versus theoretical standard error
\item 2SD method versus traditional confidence interval
\item Understand that $n > 30$ ensures the Central Limit Theorem holds
\item Understand the relationship between hypothesis testing and confidence intervals
\item Understand that not all outcomes of a random experiment are equally likely
\item Translating a research question/scenario into a null model
\item Representing null models both in words and in symbols
\end{enumerate}
\end{document}
